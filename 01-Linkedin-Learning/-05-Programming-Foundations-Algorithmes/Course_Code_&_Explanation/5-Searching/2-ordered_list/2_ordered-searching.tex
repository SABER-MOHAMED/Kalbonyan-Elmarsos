With a sorted list, we can perform a type of search is called a binary search. So, let's 
imagine we had a list of sorted numbers, and the number that we're searching for is the number
 41. To perform the binary search, we start off with two indexes at the beginning and end of 
 the list, and then we calculate the midpoint of the list, rounded down in case of an uneven 
 division, and then we check to see if that value at the midpoint is the value that we want, 
 and if it is, then great, we return that index. Now, if the number at that index is less than
  the one that we're searching for, we know that we can ignore all the numbers below that 
  index, and conversely if that number is larger than the one we're looking for, we can ignore
   all the numbers above that index. So, now we calculate the new midpoint by advancing, in 
   this case, the lower index up the middle we just calculated, and then we compute the new 
   midpoint from there. So now, the lower becomes three, the upper stays where it is, and then 
   new midpoint becomes five, after rounding down the division. And this process repeats until 
   we found the number which, in our example, is at index number five. 