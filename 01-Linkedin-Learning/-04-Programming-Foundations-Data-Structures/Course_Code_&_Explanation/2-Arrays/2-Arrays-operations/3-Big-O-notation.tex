Big-O notation 
  Notation used to describe the performance or complexity of an algorithm 

        classifies performance as the input size grows 
        "O" indicate the order of operation: time scale to perform an operation 
        It usually describes the worst case scenario of how long it takes to perform a given operation.
    Many algorithms and data structures have more than one O 
        iNserting data, searching for data, deleting data, etc.


    ---------------------------------------------------------------------
    | O(1)      | constant time | looking up a single element in array  or define if the number is even or odd
    | O(log n)  | logarithmic   | finding an item in a sorted array with a binary search 
    | O(n)      | linear time   | searching an unsorted array for a  specific value 
    | O(n log n)| Log-Linear    | complex sorting algorithm like heap sort or merge sort 
    | O(n^2)    | quadratic     | simple sorting algorithm such as bubble sort, selection sort and insertion sort

Array operations 
  calculate item index: 
    O(1)
    consistent duration of algorithms in same time (or space) 
    regardless of size of the input

  insert or delete items at beginning:
    O(n)
  insert or delete items in middle:
    O(n)

    because the remaining items must be moved to their new memory locations. 

  insert or delete item at end: 
  O(1)