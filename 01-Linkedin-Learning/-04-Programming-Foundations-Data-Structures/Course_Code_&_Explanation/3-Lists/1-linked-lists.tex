linked list:
  the elements of a linked list are not stored at contiguous locations. 
  Instead, we link the elements using pointers.
  
  is a linear collection of data elements  called nodes
  contain reference to the next node in the list
  Hold whatever data the application needs

  node: 
    contains data and a pointer to the next node
    the first item you add to the list called the head

  Operation on linked list
    
    1-Add
      To add an item, it's easiest to insert at the back or front of the list.
      When inserting at the front, we can initialize the new node with the appropriate data and set the new node's next pointer to point to the first node.
      When inserting at the back of the list, we take the last node of the list and set its next pointer to our new node.

    2- Access 
      To access an item, we don't have to have an index like we do with an array.
      This means we have to follow the pointers until we find the item we want to access.

    3- Delete
      To delete an item, we first have to find the item and then,
      update the next pointer of the node preceding and following that node.

    4- searching 
    to search for an item We have to traverse through the entire list to find anode with a specific value 
    or find out that the data does not even exist in the list.

    5- Insert 
    Let's say we wanted to add a train car somewhere else in the list.
    We'd have to follow the pointers to that specific place and then update the pointers 
    so that the previous train car points to our new train car and our new train car points to the next element.


  singly linked list:
    each item has point to the next item in the list 

  Doubly linked list:
    each item in the list has two pointers to the next and previous element 
  
  Advantages of linked list:
    1-inserting and deleting 
      elements can be easily [inserted] and [removed] with O(1) performance 
      This is because linked lists have next pointers and do not need to be stored contiguously in memory
      underlying memory doesn't need to be reorganized 
    
  Cons:
    Access
      Can't do constant-time random item access
      Item lookup[Access] is linear in time complexity (O(n))
      because items don't have index or even keywords

    Updating
      take O(n) time 
      because we need to find the appropriate node and then update its value.

    Searching and deleting
      take linear, or O of N time in the worst case, 
      because we have to search for the element in order to find it, access it, and delete it

  Sorting:
    Merge sort is often preferred for sorting a linked list. 
    Other algorithms, such as quick sort and heap sort are not ideal because linked lists have slow, random access performance. 

    In random access, we should be able to say an index and get the item at that slot immediately, like in array.